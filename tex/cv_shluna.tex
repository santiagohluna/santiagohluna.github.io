\documentclass[a4paper,10pt]{resume}
\usepackage{ebgaramond}
\usepackage{hyperref}
\hypersetup{colorlinks=true,linkcolor=blue,urlcolor=blue}
\name{Santiago Luna}
\address{Físico - Científico de Datos}
\address{Buenos Aires, Argentina}
\address{\href{mailto:santiagohluna@gmail.com}{santiagohluna@gmail.com} \\ \href{https://www.linkedin.com/in/shluna}{LinkedIn} \\ \href{https://github.com/santiagohluna}{GitHub} \\ \href{https://github.com/santiagohluna/portfolio}{Portfolio}}
\begin{document}
\begin{rSection}{Resumen profesional}
Físico con doctorado y experiencia en docencia e investigación, especializado en análisis y modelado de datos. Interesado en aplicar técnicas de machine learning, estadística y simulación a problemas complejos en ciencia e industria.
\end{rSection}
\begin{rSection}{Educación}
\begin{rSubsection}{Doctor en Física}{2015 --- 2020}{Universidad Nacional de Rosario}{}
\item Investigación en modelado y simulación de sistemas físicos complejos, con énfasis en análisis computacional y procesamiento de datos observacionales.
\end{rSubsection}
\begin{rSubsection}{Licenciado en Física}{2008 --- 2015}{Universidad Nacional de Rosario}{}
\item Formación en física teórica, experimental y computacional.
\end{rSubsection}
\end{rSection}
\begin{rSection}{Experiencia profesional}
\begin{rSubsection}{Docente e Investigador}{2018 --- Presente}{Universidad Nacional de La Matanza}{}
\item Dictado de clases y desarrollo de material didáctico en física y matemáticas.
\item Investigación aplicada en simulación y análisis de datos físicos.
\item Desarrollo de herramientas computacionales para el análisis y visualización de resultados.
\end{rSubsection}
\end{rSection}
\begin{rSection}{Proyectos personales}
\begin{rSubsection}{Predicción de la aceleración de la gravedad con Machine Learning}{2025}{Desarrollador e investigador}{}
\item Proyecto educativo-experimental que utiliza modelos de regresión para estimar la aceleración de la gravedad a partir de datos obtenidos de un péndulo simple.
\item Tecnologías: Python, scikit-learn, pandas, matplotlib
\item Repositorio: \href{https://github.com/santiagohluna/pendulo-ml}{https://github.com/santiagohluna/pendulo-ml}
\end{rSubsection}
\begin{rSubsection}{Sistema de gestión de activos y mantenimiento}{2025}{Desarrollador full stack}{}
\item Aplicación web basada en Flask y SQLite para gestionar activos, sectores y pedidos de mantenimiento. Incluye operaciones CRUD completas.
\item Tecnologías: Python, Flask, SQLite, HTML, CSS, JavaScript
\item Repositorio: \href{https://github.com/santiagohluna/gestion-mantenimiento}{https://github.com/santiagohluna/gestion-mantenimiento}
\end{rSubsection}
\begin{rSubsection}{Dashboard de ventas interactivo}{2025}{Analista de datos}{}
\item Desarrollo de un panel interactivo en Power BI para análisis de ventas por categoría, región, mes y medio de pago.
\item Tecnologías: Power BI, Python, pandas
\item Repositorio: \href{https://github.com/santiagohluna/ventas-bi}{https://github.com/santiagohluna/ventas-bi}
\end{rSubsection}
\end{rSection}
\begin{rSection}{Publicaciones destacadas}
\item Modelado de sistemas dinámicos con métodos de aprendizaje automático, Congreso Argentino de Física Computacional, 2021. \href{https://example.com/publicacion1}{Enlace}
\end{rSection}
\begin{rSection}{Habilidades}
\textbf{Técnicas:} Python, Fortran, Java, SQL, Flask, Django, Power BI, Git, GitHub, Visual Studio, JupyterLab, SQLite, MySQL, SQL Server, Dataverse, Control de versiones, Desarrollo web, Análisis estadístico, Machine Learning.
\textbf{Blandas:} Comunicación efectiva, Trabajo en equipo, Pensamiento analítico, Docencia y mentoría, Resolución de problemas.
\end{rSection}
\begin{rSection}{Certificaciones y cursos}
\item Diplomatura en Administración Rural, UTN FRRq, 2025. \href{https://example.com/certificado-administracion-rural}{Certificado}
\item Curso de Business Intelligence con Power BI, Plataforma educativa online, 2025. \href{https://example.com/certificado-power-bi}{Certificado}
\item Curso introductorio de Machine Learning, Coursera / Universidad de Stanford, 2024. \href{https://www.coursera.org/account/accomplishments/certificate/EXAMPLE}{Certificado}
\end{rSection}
\begin{rSection}{Idiomas}
Español (Nativo), Inglés (Avanzado)
\end{rSection}
\begin{rSection}{Intereses}
Ciencia de datos, Física computacional, Machine Learning, Business Intelligence, Optimización y simulación numérica
\end{rSection}
\end{document}